\chapter{Compiler}
\label{chap-compiler}

The compiler mainly consists of the pre-compiler mentioned in
\refSec{sec-evaluator-implementation-general-strategy}.  However,
there are minor distinctions between the use of the compiler in
\texttt{eval}, \texttt{compile}, and \texttt{compile-file}.

\section{\texttt{compile-file}}

The result of the pre-compilation is written to a file as expected.

When \texttt{compile-file} encounters an \texttt{eval-when} form as a
top-level form, it behaves as follows:

\begin{itemize}
\item If the situation \texttt{:compile-toplevel} or \texttt{compile}
  was given, then \texttt{compile-file} calls \texttt{eval}, passing
  it the body form of the \texttt{eval-when} form.
\item If the situation \texttt{:load-toplevel} or \texttt{load} was
  given, then \texttt{compile-file} transforms the body form as usual,
  inserting the result at the top-level of the transformed file
  contents so that it will be interpreted when the file is loaded.
\end{itemize}

