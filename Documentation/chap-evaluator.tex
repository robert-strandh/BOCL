\chapter{Evaluator implementation}
\label{chap-evaluator-implementation}

\section{General strategy}
\label{sec-evaluator-implementation-general-strategy}

The evaluator is implemented as an interpreter written in
\clanguage{}.  However, we do not interpret the \commonlisp{} code
directly.  Instead, it is first \emph{minimally compiled} using a
pre-compiler, resulting in a program using a notation that is based on
S-expressions, but that is not conforming \commonlisp{}.  There are
several reasons for the use of this technique:

\begin{itemize}
\item Minimal compilation is a term used in the \commonlisp{}
  standard, and its minimal requirements include expansion of all
  macros and compiler macros.  Minimal compilation is a requirement of
  the file compiler, so this notation can be used as a FASL notation
  as well as an intermediate notation for the interpreter.
\item We reduce the number of different kinds of forms that the
  interpreter must handle, thereby allowing a simpler interpreter,
  which is important since it is written in \clanguage{}.
\item The pre-compiler can be written in \commonlisp{}, or, rather, in
  the pre-compiled notation that is acceptable to the interpreter.
  Since the code of the pre-compiler must be interpreted, this process
  is likely sluggish, but it makes the resulting code faster to
  interpret.  We think this technique will be a good compromise
  between the performance and the maintainability of \sysname{}.
  A first version of the pre-compiler could be written in full
  \commonlisp{} and then be pre-compiled by itself to obtain the
  version intended to be maintained, and included in the distribution
  of \sysname{}.
\end{itemize}

In the remainder of this chapter, we discuss the details of the
pre-compiled notation and the details of the interpreter for this
notation.

\section{Pre-compiled S-expression notation}

\subsection{Expansion of all macros and symbol macros}

As required by the \commonlisp{} standard, minimally compiled code
must have all its macros and symbol macros expanded.  \sysname{} takes
advantage of the freedom given by the \commonlisp{} standard to never
expand compiler macros, though application code must obviously be
allowed to define compiler macros with compiler macro functions being
correctly generated.

\subsection{Elimination of implicit \texttt{progn}}

All implicit \texttt{progn}s are replaced by an explicit
\texttt{progn}, so that the interpreter has only one special operator
where the sequence of sub-forms needs to be iterated over.

\subsection{Passing arguments and returning values}

All functions are called with a list of arguments, and all functions
return a list of return values.  Using lists greatly simplifies the
code and makes it much easier to implement standard \commonlisp{}
functions in \clanguage{}

\subsection{Pre-processing ordinary lambda lists}
\label{pre-processing-ordinary-lambda-lists}

Ordinary lambda lists have a fairly complex syntax.  In particular,
there might be initialization forms for \texttt{\&optional} and
\texttt{\&key} parameters, and they allow for variables to be
introduced using \texttt{\&aux}.  We turn all such lambda lists into
one with a single parameter that holds the list of all arguments upon
function entry.  Explicit code in the beginning of the function body
parses the list of arguments and binds the variables used in the
original function body.

\subsection{Elimination of bindings of special variables}

Special variables can be bound in several special forms and in lambda
lists.  We eliminate such bindings in favor of the use of
\texttt{progv} which turns the special variables into symbols, so that
they are treated as any other data.  When a special form such as
\texttt{let} or \texttt{let*} binds a special variable, the form may
need to be split into several sub-forms, with a \texttt{progv} in
between.

\subsection{Elimination of all declarations}

The \commonlisp{} standard gives considerable freedom to a conforming
implementation to ignore most declarations.  The exceptions are
\texttt{declaration}, \texttt{notinline}, \texttt{safety}, and
\texttt{special}.  The pre-compiler eliminates the others directly.
The remaining ones are handled as follows:

\begin{itemize}
\item The \texttt{declaration} declaration is processed by the
  pre-compiler together with the associated declarations using it.
  In the output, they are then eliminated.
\item The \texttt{notinline} declaration is eliminated by the
  pre-compiler, which is possible since \sysname{} does not inline
  application functions at all.  The standard allows for an
  implementation that does not inline and that does not use compiler
  macros to ignore this declaration.
\item The \texttt{safety} is eliminated by the pre-compiler.  The
  standard allows us to do this if all code is processed as if the
  highest safety value were used.
\item The \texttt{special} declaration can be eliminated by the
  pre-compiler since all bindings of special variables have been
  replaced by the use of the \texttt{progv} special form. 
\end{itemize}

\subsection{Elimination of \texttt{locally}}

Since the sole purpose of \texttt{locally} is to introduce local
declarations, and declarations have already been eliminated, there is
no need for \texttt{locally}, so \texttt{locally} is eliminated as well.

\subsection{Representing the dynamic environment}

The dynamic environment is represented as an ordinary heap-allocated
\commonlisp{} list of \emph{entries}.  The following types of entries
exist:

\begin{itemize}
\item \texttt{block/return}
\item \texttt{tagbody/go}
\item \texttt{unwind-protect}
\item \texttt{catch}
\item special-variable binding.
\item restart.
\end{itemize}

The interpreter handles many operators that manipulate the dynamic
environment as function calls, for example \texttt{block},
\texttt{return-from}, \texttt{tagbody} , \texttt{go} , \texttt{progv},
\texttt{catch}, \texttt{throw}.  This organization greatly simplifies
the interpreter.

\subsection{\texttt{block} and \texttt{return-from}}

The block name of each \texttt{block} and \texttt{return-from} is
replaced by an uninterned symbol that is unique to the compilation
unit.  The interpreter uses this symbol as the name of a lexical
variable that is introduced into the lexical environment by the
interpreter when it interprets the \texttt{block} special form.  The
value of the lexical variable is a unique object created by the
equivalent of the execution of the form \texttt{(list nil)}.  This
value is also used to create a \texttt{block/return-from} entry in the
dynamic environment.  The interpreter interprets the body%
\footnote{The body consists of a single form since implicit
  \texttt{progn}s have been eliminated.}
of the \texttt{block} special form using the new lexical and dynamic
environment.

When
\texttt{return-from} is interpreted, the lexical environment is
consulted for the value of this variable, and the dynamic environment
is searched for a \texttt{block/return-from} entry with that value
stored in it in order to perform a non-local control transfer.

\subsection{\texttt{tagbody} and \texttt{go}}

Perhaps \texttt{tagbody} and \texttt{go} are the special operators
that undergo the most significant transformation.  A \texttt{tagbody}
form is transformed into \texttt{(tagbody} \textit{symbol
  form}\texttt{)} where \textit{symbol} is an uninterned symbol
playing the same role as the uninterned symbol in \texttt{block} and
\texttt{return-from}.  The \textit{form} is typically a \texttt{progn}
where the body consists of all the \emph{statements} (i.e., excluding
the tags) of the original \texttt{tagbody} form.

A \texttt{go} form is transformed into \texttt{(go} \textit{symbol
  form}\texttt{)}, where \textit{symbol} is identical to the one in
the corresponding transformed \texttt{tagbody} form.  The
\textit{form} is again often a \texttt{progn} form consisting of all
the statements in the original \texttt{tagbody} form that follow the
\texttt{go} tag of the original \texttt{go} form.

For example, consider an original form such as \texttt{(tagbody}
\textit{statement1 tag1 statement2 statement3 tag2
  statement4}\texttt{)}.  This form is transformed into
\texttt{(tagbody} \textit{symbol} \texttt{(progn} \textit{statement1
  statement2 statement3 statement4}\texttt{))}.  An original form
\texttt{(go} \textit{tag1}\texttt{)} is transformed into
\texttt{(go} \textit{symbol} \texttt{(progn} \textit{statement2
  statement3 statement4}\texttt{))}, and 
an original form
\texttt{(go} \textit{tag2}\texttt{)} is transformed into
\texttt{(go} \textit{symbol} \textit{statement4}\texttt{)}.

When the interpreter encounters a transformed \texttt{tagbody} form,
it creates a variable in lexical environment with the name
\textit{symbol} and with a value being the result of evaluating
\texttt{(list nil)} just as with \texttt{block} and
\texttt{return-from}.  The interpreter also creates a
\texttt{tagbody/go} entry in the dynamic environment containing the
unique value.  The body of the \texttt{tagbody} form is interpreted in
the modified lexical and dynamic environments.  When the interpreter
encounters a transformed \texttt{go} form, it searches the dynamic
environment for a \texttt{tagbody/go} entry with the same value as
that of the lexical variable.  It then performs a non-local control
transfer to the place where the body of the \texttt{tagbody} form is
interpreted, sending the form there to be interpreted.

\subsection{\texttt{catch} and \texttt{throw}}

The only change made to these special forms is that the implicit
\texttt{progn} of \texttt{catch} has been replaced by an explicit
\texttt{progn} form in case there is more than one form in the body of
the \texttt{catch} form.

\subsection{\texttt{flet} and \texttt{labels}}

The implicit \texttt{progn} of the body of an \texttt{flet} or a
\texttt{labels} form is eliminated as usual.  The lambda lists of the
functions being defined are simplified as indicated in
\refSec{pre-processing-ordinary-lambda-lists}.  And the bodies of the
functions being defined are pre-processed as usual.

\subsection{\texttt{eval-when}}

All \texttt{eval-when} forms are eliminated in the minimally compiled
code.  Such forms are processed according to which entry point into
the compiler was used, as explained in \refChap{chap-compiler}.
