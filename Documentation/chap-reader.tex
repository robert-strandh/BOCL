\chapter{Reader}

A reader is required in order to read \commonlisp{} code to turn the
core functionality written in \clanguage{} into a fully conforming
\commonlisp{} implementation.  The reader used to read most code is
Eclector, but we need a simple reader written in \clanguage{} in order
to read the source code of Eclector.  However, this simple reader does
not need to have all the features of a complete \commonlisp{} reader,
so it differs from a conforming reader as follows:

\begin{itemize}
\item The input base for reading numbers is set to 10 and can not be
  changed.
\item Only the reader macros needed to read Eclector are implemented,
  and they are not implemented as a dispatch through a readtable.
  Instead their functions are hard coded into the simple reader.
\item The accessor \texttt{readtable-case} is not implemented, and the
  simple reader behaves as if the readtable case were
  \texttt{:upcase}.
\item The simple reader assumes that the input is well formed, so it
  makes no attempt to check incorrect input, and it has no provisions
  for signaling errors. 
\end{itemize}

